\documentclass[11pt]{article}

% if you need to pass options to natbib, use, e.g.:
%     \PassOptionsToPackage{numbers, compress}{natbib}
% before loading neurips_2019

% ready for submission
% \usepackage{neurips_2019}

% to compile a preprint version, e.g., for submission to arXiv, add add the
% [preprint] option:
%     \usepackage[preprint]{neurips_2019}

% to compile a camera-ready version, add the [final] option, e.g.:
\usepackage[final]{../neurips_2019}

% to avoid loading the natbib package, add option nonatbib:
%     \usepackage[nonatbib]{neurips_2019}

\usepackage[utf8]{inputenc} % allow utf-8 input
\usepackage[T1]{fontenc}    % use 8-bit T1 fonts
\usepackage{hyperref}       % hyperlinks
\usepackage{url}            % simple URL typesetting
\usepackage{booktabs}       % professional-quality tables
\usepackage{amsfonts}       % blackboard math symbols
\usepackage{nicefrac}       % compact symbols for 1/2, etc.
\usepackage{microtype}      % microtypography
\usepackage{enumitem}
\title{COMP 755 Midway Report}

\author{%
    Nick Rewkowski
    \and
    D. Ben Knoble
    \and
    Conny Lu
    \and
    Dongxu Zhao
    \and
    Hanpeng Wang
}

\begin{document}
\maketitle

% PROJECT IDEA & SUMMARY

%TODO
%What project idea are you pursuing? Clearly state the task that you are performing, and the goals of the methods you will explore (e.g. better accuracy, sharper samples, quicker computation, etc.).

%\section{Midway Report}


\section{Introduction}

%%%%% Problem/Applications/Challenge
Over the last few years, virtual reality (VR) and augmented reality (AR)
technologies demonstrate the potential to effect the way we live and work.
Understanding user behavior in VR/AR is very important for both creating
content and adapting to different user models. In this paper, we use
knowledge of where the user looks to detect objects and
understand various user behaviors such as
\begin{inlist}
\item looking at an object,
\item being interested in the object (e.g., predicted trajectory), or
\item
    paying attention to a certain detail (e.g., a training surgeon carefully
    assembling the surgical bench).
\end{inlist}

%%%%% Background of object detection + our work
Currently, object detection and eye tracking are both well-understood problems.
State-of-the-art methods unanimously employ variants of convolutional neural
networks (CNNs). A number of object detectors~\cite{redmon2016you, liu2016ssd,
lin2017focal, girshick2014rich, girshick2015fast, ren2015faster, he2017mask}
achieve incredible results on large-scale benchmarks. These networks are
designed to recognize low-resolution data in a large search space, relying on a
deep neural network with a high memory capacity requirement. In order to achieve
a more accurate and performant version on VR/AR platforms, we propose a novel
gaze-guided object detection method that first tracks gaze with a built-in
eye-tracker in Magicleap and then segments and detects the focused object with a
smaller neural network.

To decrease time cost as well as collection difficulty, we generate a huge
synthetic dataset including both eye images and scene images in an AR
environment. Game engines, which are capable of aligning real and virtual worlds
with markers, can provide the ground truth of focused point without the need for
manual labelling. From this, we can move on to more complex scenarios and use
such labelling for tasks such as generation of behavioral graphs or other
modern trends. Some use cases that motivated this project include:

\begin{description}
    \item[Adaptive Impaired-user Keyboards]
        If some symbols on a virtual keyboard are harder to use than others, or
        are used frequently together, re-arranging the layout to correspond to
        the user's needs may improve typing experience. Since some of these are
        joystick-input based and therefore further slow user's typing rate, the
        user's gaze can identify what symbol they are typing and aid the user in
        selecting it. Some users with mobility impairments may also want to
        select an item by blinking, speaking, or other means. Improvements in
        this area (such as via faster object detection based on gaze, as in this
        paper) could prove important because different disabilities require
        different customized keyboards for comfortable typing.
    \item[Surgical Training]
        Surgical students learn from the difference between the steps they take
        and the correct steps. For example, if a student's job is to organize
        and clean tools for a surgery, we can use eye-tracking and logs from the
        real-time application to generate a behavior graph (push-down
        automata-style) and determine if they are missing important steps or
        glossing over details.
    \item[Dynamic Robot AI]
        AI assisted by object recognition and behavioral understanding can
        better determine how to respond to user actions, such as in the case of
        a robot guide dog pulling users towards their object of interest. For
        example, the dog should realize when a VR user is too close to a wall
        and try to guide them away.
\end{description}

%%%%% Contribution
Our main contributions can be summarized as follows:

\begin{itemize}
    \item
        We automatically generate a ground truth object of interest with a
        real-time AR simulation in a game engine allowing raycasting
        (using Python, C++, and the Unreal Engine).
    \item
        We introduce a more efficient method of integrating object detection
        with eye tracking that reduces the amount of computation.
    \item
        We evaluate both detection methods with and without eye tracking and
        demonstrate that our method compares favorably against detection methods
        without eye tracking.
    % perhaps we only discuss this, given the timeline?
    % \item
    %     We apply our gaze-guided object detection on specific AR applications
    %     for user behavior understanding.
\end{itemize}


\section{Related Work}
%nick knows nothing about this. nick doesn't pay attention in class either so
%doesn't know where to look. also neurips, iccv, cvpr, etc. are filled with fake
%papers so nick's not confident in their results anyway

\subsection{Eye tracking}

Eye tracking is defined as the process of identifying the line of sight for each
eye of a human user at a single instant~\cite{kim2019nvgaze}. In this work, we
focus on vision-based eye tracking, which comprises mainly three methods:
feature-based tracking, model-based tracking, and appearance-based tracking.
Feature-based tracking localizes key eye features, such as the iris center and
eye corner, and maps them to corresponding gaze coordinates on the
screen~\cite{sesma2012evaluation, torricelli2008neural}. Model-based
methods~\cite{wood20163d, wang2017real} estimate eye gaze using a geometrical
model of the human eye to simulate the structure and function of the human
vision system. The space of state-of-the-art appearance-based methods is
occupied by artificial neural networks. A huge collection of eye images are used
to train a neural network to output the final gaze~\cite{zhang2015appearance,
schneider2014manifold, sugano2014learning}. A hybrid of those three methods has
also been explored in~\cite{kim2019nvgaze, wang2018hierarchical}.

Vision-based tracking is important for real-world real-time applications, where
the information available for decision-making is likely to consist only of
representations of the eye. Our focus is not to further improve eye-tracking
accuracy but to explore the integration of eye-tracking information (such as
might be gathered by the kinds of vision-based trackers described above) into
image recognition systems. We thus explored primarily the use of a novel kind of
synthetic dataset (see Section~\ref{S:synthetic-dataset} and
Section~\ref{S:synthetic-generation}) that simulates gaze and computes
automatically the coordinates and objects gazed at.

\subsection{Image Recognition}

Image recognition is fundamentally the problem of analysing image data to
recognize a part of the image. Usual tasks include recognizing or classifying
certain objects or their boundaries, including animals, vehicles, and even
textual characters. Example datasets include the MNIST database of handwritten
digits~\cite{Lecun_1998} and the COCO (common objects in context)
dataset~\cite{COCO}. One core problem in image recognition tasks is that even
low-resolution images are high-dimensional inputs, since each pixel consists of
3 color channels (red, green, and blue). For a 480p image, or \(640 \times 480\)
pixels, this is \(640 \times 480 \times 3 = 921600\) features per image.
Convolutional neural networks such as in~\cite{Lecun_1998}
and~\cite{Krizhevsky_2017} help reduce the impact of this problem with
convolutions and other techniques.

Due to advances in deep learning, there have been a notable number of image
recognition models~\cite{simonyan2014very, szegedy2015going,
szegedy2016rethinking, he2016deep, szegedy2017inception} which achieve similar
and strong performance. Among these models, we chose
Inception-v3~\cite{szegedy2016rethinking} for two reasons. First, it has high
computational efficiency and low parameter count with the use of factorization
in convolutions and aggressive regularization, making it suitable for uses
in mobile vision and real-time applications. Second, it has been proven that
Inception-v3 can perform well even on low-resolution images for small objects
in~\cite{szegedy2016rethinking}, which is essential in our application.

Inception-v3 is a successor of Inception-v1~\cite{szegedy2015going} which
further improves the computational efficiency of the Inception module. It first
factorizes convolutions with larger spatial filters into smaller convolutions,
for example, replacing one $5 \times 5$ convolution with two layers of $3 \times
3$ convolution. Further, it spatially factorizes convolutions into asymmetric
convolutions by replacing any $n \times n$ convolution by a $1 \times n$
convolution followed by a $n \times 1$ convolution. In addition, it adds tweaks
including regularization with batch-normalized auxiliary classifiers and label
smoothing, which enables training high quality networks using only a modest
sized training set. Inception uses input dimensions of \(299 \times 299\), which
results in a feature-space of dimension 268203.

While investigating the integration of eye-tracking data with image recognition,
we focussed on maintaining similar or better recognition performance while
improving computational complexity and speed. We hypothesized that some methods
of integrating eye-tracking information result in stronger gains than others,
and these gains are important to real-time applications, which are heavily
constrained.

\subsection{Object detection}

Current object detection approaches can be categorized into one-stage
approaches~\cite{redmon2016you, liu2016ssd, lin2017focal} and two-stage
approaches~\cite{girshick2014rich, girshick2015fast, ren2015faster, he2017mask}.
Two-stage detectors first generate several proposals with high probability of
containing an object, and then use two sub-networks to classify each
proposal and regress their specific positions separately. On the other hand,
one-stage methods detect objects without using a region proposal network (RPN)
for a simpler network architecture and shorter inference time. For more details
please refer to the latest overview\cite{liu2020deep}.

When considering deployment in several resource-constrained platforms, such as
VR/AR HMD with memory limitations and high frame rate requirements, both methods
are still far from practical use. To solve this issue, we integrate object
detection with eye tracking to accelerate performance and reduce network
capacity by decreasing the search space. Some related works~\cite{toyama2012gaze,
ishiguro2010aided, bonino2009blueprint} also combine these two tasks, but they
either do not provide an evaluation of the benefits of using eye tracking or
require a collection of all exhibits with static backgrounds.

\subsection{Synthetic eye dataset}\label{S:synthetic-dataset}

Previous research showed that synthetic eye dataset relieves the eye-tracking
research community from manual data collection and data labeling.
\emph{NVGaze}~\cite{kim2019nvgaze} dataset, for example, renders 2 million
infrared images of eyes at \(1280 \times 960\) resolution using
anatomically-informed eye and face models.
\emph{SynthesEyes}~\cite{wood2015rendering} samples 11.4 thousand photorealistic
eye images with a wide range of head poses, gaze directions, and illumination
conditions. As a follow-up work, \emph{UnityEyes}~\cite{wood2016learning}
provides a 200-times faster solution to generating large numbers of images.


We explore our novel dataset generation in Section~\ref{S:synthetic-generation}.


\vspace{-1em}
\section{Methods}
The system used for this project is a pipeline consisting of synthetic image data collection, automatic creation of ground truth labels, intermediate image processing, training, and finally, image recognition during a realtime application.
%test
\subsection{Synthetic Data}
\vspace{-1em}
While there are existing datasets of eyes for eye-tracking research, their purpose is significantly different from ours and generally only include labels such as eye rotation; to our knowledge, such a dataset does not exist which labels the object of interest. Thus, for this project, we must generate synthetic data ourselves. We require synthetic data as we would not be able to easily record real data in this scenario given both the tediousness of doing so and our quarantine.
\newline\indent Much like how \textit{UnityEyes} renders eye images very quickly using a realtime game engine, we use Unreal Engine 4 (UE4) to render (1) camera images of the eyes, (2) camera images from the front-facing camera approximate the user's entire field of view, and (3) ground truth labels for what object the user is looking at. We choose UE4 for a few reasons:
\begin{itemize}[leftmargin=*,noitemsep]
    \item It has a strong physically-based rendering (PBR) system allowing for much more realistic materials to be used than Unity and with better performance
    \item It natively supports ray-tracing and cone-tracing \cite{???} allowing for more accurate eye reflections (which heavily impact eye-tracking quality in real life) and more accurate environmental reflections in general
    \item It supports C++, allowing interaction with image recognition libraries such as NVIDIA's
    \item NVIDIA, a major name in deep neural networks in general, has much stronger experimental features for UE4 such as CUDA, geometry interaction, etc.
\end{itemize}
We use free, high-quality architectural visualization (archviz) scenes for UE4 such as ``Realistic Rendering'' which contains many environmental objects with clear, intuitive boundaries for the rendering of the user viewpoint and labelling of focused objects. The objects in this scene include vases, books, furniture, paintings, and plants. They are labelled as such in the default scene. For the eye-facing camera images, we use Adobe Mixamo's character collection to handle different eye and body shapes and materials. These characters most importantly include blendshapes allowing us to make them blink naturally during data collection, which would also affect real data. The limitation of these characters is that their materials such as are not highly detailed and some detailed facial features like accurate wrinkles on the eyelids are missing. Examples of what the images generated look like can be found in the supplemental material.
\subsection{Automatically Generating Ground Truth Labels}
An important feature of realtime game engines is native and efficient support for raycasts, which are functions that shoot rays from one point to another, returning all objects hit by the ray as well as hit location, hit normal, hit polygon, etc. These allow for the automatic deduction of what a camera is looking at. In the case of our eye-tracking setup, the MagicLeap One gives the 3D point that the users' eyes converge on, thus, we can raycast from the front-facing camera location to this point of convergence and return the first object hit in order to estimate what the user is looking at.
\newline\indent In order to generate the required dataset, we define a 3D scanpath describing the trajectory of the user's gaze that spans all of the objects in the scene and eventually causes the character's body to turn 360 degrees. For each Mixamo character, their eyes follow the scanpath and for each frame (at 60fps), we generate the eye-facing images, the front-facing camera image, and the label for what the raycast hit during that frame. It currently takes 2 minutes for the character to make the full rotation as this scene is relatively small; we plan on adding more complexity later. For future data collection, we may randomly select objects to look at in the 3D environment to avoid the manual placement of scanpaths which can be a limiting factor in larger scenes.
\subsection{Intermediate Image Processing}
This intermediate step occurs between the synthetic data generation and training because networks like Inception V3 were not intended to handle more than 1 image per label, e.g. they would not understand the relationship between the eye camera and front-facing camera images. Thus, during this step, we crop the object in focus such that the image's boundaries only span that object, allowing for a lower-res image with a single primary object, meeting our objective of simplifying the inputs to the image recognition network. For potential comparisons that we make, this step could also be used to do post-processing such as defocusing the parts of the image not being focused on (e.g. applying Gaussian blur to anything outside the object bounds) instead of cropping the object of interest. This is similar to how foveated depth-of-field works; a more recent method in AR/VR in which depth-of-field is applied automatically based on eye-tracking.
\subsection{Training \& Recognition}
The training and recognition part involves using the popular Inception V3 network to recognize the primary object in the images given to the network. Since the game engine generated the labels automatically, the ground truth labels were retrieved easily and are given to the network as with any other use of this network. Beyond that, this part is fairly standard image recognition and details of Inception V3 are better found in that network's API.
\subsection{Testing with a Realtime Application}
During a realtime MagicLeap application, if training is successful, we should only require the images from the front-facing camera and eye-tracking camera, which can be merged in the intermediate image processing step and given to Inception V3 for prediction. The 3D scene from the game engine would no longer be required and it is even possible that a different, less bulky HMD besides the MagicLeap could be used as long as it contains the necessary cameras (e.g. embedded Raspberry Pi cameras). 
\newline\indent In order to help validate the ability of the system to work on real-world camera images, in order to get the correct labels for the objects of interest, it would be easiest to align some virtual objects to the real objects in order to use raycasts for automatic labelling. This alignment can be done automatically with image marker-tracking libraries that work natively with the MagicLeap and Hololens such as OpenCV and Vuforia.

\vspace{-1em}
\section{Preliminary Results}
Much of our progress thus far has been in constructing the synthetic data collection system itself and creating the environment collection consisting of UE4, the MagicLeap API, and Inception V3, as the synthetic data collection itself is the most laborious part of the system as the image recognition itself uses a standard, easy-to-use library. In some informal tests, we did find that the cropped image had a much higher recognition confidence as compared to the image in context. A current limitation is that meeting the 60fps capture mark will take more optimization of the data collection system; at the moment, it runs at about 15fps due to the process of writing the images to disk. Rendering these sequences as videos and deconstructing them afterwards tends to be more efficient for game engines and is a likely next step for efficiency.

\vspace{-1em}
\section{Future Plans}
%Provide an outline with estimated dates of completion for the rest of the project.
\subsection{Create Dataset for Inception V3 Training}
Our very next step is to render the final training dataset to allow us to move onto the intermediate image processing step and test the results in Inception V3. This would be finished by the end of this week (by 4/13/20).

\subsection{Comparison of Image Processing Step}
Our major analytical contribution will be understanding how the overall system improves the object recognition accuracy of image recognition libraries. Thus, we want to compare 3 inputs to Inception V3: (1) no image processing, only the front-facing camera, (2) cropped image, (3) defocused image. We would expect the cropped image of only the object to be the most accurate method, defocused the 2nd best, and the full image the worst. This is the absolute minimum viable product of the project and would ideally be completed at least 1 week before the last day of classes.

\subsection{Testing on Real Data}
By the last day of classes, we hope to have tested at least some simple scenario on the real MagicLeap with real camera data.

\subsection{Test the Effectiveness of Small Objects Detection Integrated with Eye Tracking}
As stated above, our first primary future work would be verifying the effectiveness of our pipeline in object detection. It is known that small objects detection is a challenge in tracking, especially in complex and cluttered images, so we want to test the robustness of the cropping as a stretch goal.




%oh you're back
\bibliography{references}
%\clearpage
\appendix
\renewcommand{\thefigure}{SM\arabic{figure}}
\setcounter{figure}{0}
\section{Supplemental Material}\label{sec:suppl}

\end{document}    