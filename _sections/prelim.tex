\section{Preliminary Results}

Much of our progress thus far has been in constructing the synthetic data
collection system itself and creating the environment collection consisting of
UE4, the MagicLeap API, and Inception V3, as the synthetic data collection
itself is the most laborious part of the system as the image recognition itself
uses a standard, easy-to-use library. In some informal tests, we did find that
the cropped image had a much higher recognition confidence as compared to the
image in context. A current limitation is that meeting the 60fps capture mark
will take more optimization of the data collection system; at the moment, it
runs at about 15fps due to the process of writing the images to disk. Rendering
these sequences as videos and deconstructing them afterwards tends to be more
efficient for game engines and is a likely next step for efficiency.
