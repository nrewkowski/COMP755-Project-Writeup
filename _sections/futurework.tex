\section{Future Plans}

%Provide an outline with estimated dates of completion for the rest of the project.

\subsection{Create Dataset for Inception V3 Training}

Our very next step is to render the final training dataset to allow us to move
onto the intermediate image processing step and test the results in Inception
V3. This would be finished by the end of this week (by 4/13/20).

\subsection{Comparison of Image Processing Step}

Our major analytical contribution will be understanding how the overall system
improves the object recognition accuracy of image recognition libraries. Thus,
we want to compare 3 inputs to Inception V3:
\begin{inlist}
\item unprocessed images, i.e., using only the front-facing camera;
\item cropped images; and
\item defocused images.
\end{inlist}
We would expect the cropped image of only the object to be the most accurate
method, defocused the 2nd best, and the full image the worst. This is the
absolute minimum viable product of the project and would ideally be completed at
least 1 week before the last day of classes.

\subsection{Testing on Real Data}

By the last day of classes, we hope to have tested at least some simple scenario
on the real MagicLeap with real camera data.

\subsection{Test the Effectiveness of Small Objects Detection Integrated with Eye Tracking}

As stated above, our first primary future work would be verifying the
effectiveness of our pipeline in object detection. It is known that small
objects detection is a challenge in tracking, especially in complex and
cluttered images, so we want to test the robustness of the cropping as a stretch
goal.
